%\VignetteIndexEntry{wavClusteR}
%\VignettePackage{wavClusteR}

\documentclass{article}

\usepackage{Sweave}
\usepackage[a4paper]{geometry}
\usepackage{hyperref,graphicx}
\usepackage{color}

%\SweaveOpts{keep.source=TRUE,eps=TRUE,width=8,height=8.5} 
\newcommand{\Robject}[1]{\texttt{#1}}
\newcommand{\ext}[1]{\texttt{#1}}
\newcommand{\Rpackage}[1]{\textit{#1}}
\newcommand{\Rclass}[1]{\textit{#1}}
\newcommand{\Rfunction}[1]{{\texttt{#1}}}
\newcommand{\disc}[1]{\color{red} #1 \color{black}}
\renewcommand{\floatpagefraction}{0.7}	
\setkeys{Gin}{width=0.8\textwidth}
  
\author{Federico Comoglio and Cem Sievers\\
Department of Biosystems Science and Engineering\\
ETH Z\"urich, Basel, Switzerland\\
\texttt{federico.comoglio@bsse.ethz.ch}\\
\texttt{cem.sievers@bsse.ethz.ch}}

\title{\textsf{\textbf{The \Rpackage{wavClusteR} package \\{\large (version 2.0)}}}}

\begin{document}

\maketitle
\begin{abstract}
Different recently developed next-generation sequencing based methods (e.g. PAR-CLIP or Bisulphite sequencing) specifically induce nucleotide substitutions within the short reads with respect to the reference genome. This package provides functions for the analysis of the data obtained by such methods - with a major focus on PAR-CLIP - and exploits the experimentally induced substitutions in order to identify high confidence signals, such as RNA-binding sites, in the data. The workflow consists of two steps; (i) the estimation of a non-parametri two-component mixture model, identifying substitution frequencies most affected by the experimental procedure; (ii) a binding sites (clusters) identification algorithm which resolves clusters at high resolution. Key functions support multicore computing, if available. For a detailed description of the method see \cite{cemo, cf}.
\end{abstract}

\tableofcontents

%--------------------------------------------------
\section{Preparing the input} 
%--------------------------------------------------

Starting with a fastq file, a commonly used short read format, the short reads should be aligned to the reference genome using a short read aligner, e.g. Bowtie \cite{Langmead:2009p2762}. The output file (e.g. in SAM format) should then be converted to BAM format (e.g. using \ext{samtools view}, \url{http://samtools.sourceforge.net/samtools.shtml}) and sorted (e.g. using \ext{samtools sort}). Since \Rpackage{wavClusteR} requires an indexed BAM file containing the short read alignments, an index file (\texttt{.bai}) should be generated from the sorted BAM file (see, e.g. \ext{samtools index}). \newline
The following code provides an example of the steps described above using the samtools toolkit. The first line is pseudo code. Please replace it with the aligner specific syntax.

\begin{verbatim}

	ALIGN: sample.fastq -> sample.sam
	
	CONVERT: samtools view -b -S sample.sam  -o sample.bam
	
	SORT: samtools sort sample.bam sample_sorted
	
	INDEXING: samtools index sample_sorted.bam

\end{verbatim}

%%--------------------------------------------------
\subsection{Example dataset} 
%%--------------------------------------------------
In this vignette, we consider as an example a chunk of a published Argonaute 2 (AGO2) PAR-CLIP data set obtained from human HEK293 cells \cite{Kishore:2011p4559}. This chunk contains reads mapping to chromosome X in the interval: 23996166 - 24023263. This data set is provided in \Rpackage{wavClusteR} 2.0.
%
%%--------------------------------------------------
\subsection{Importing short reads into the R session} 
%%--------------------------------------------------
%
An indexed BAM file can be loaded into the R session using the \Rfunction{readSortedBam} function. This calls \Rfunction{scanBam} from \Rpackage{Rsamtools}~\cite{rsamtools} and extracts the mismatch MD field and the read sequence from the BAM file, returning a \Robject{GRanges} object.
%
\begin{Schunk}
\begin{Sinput}
> library(wavClusteR)
> filename <- system.file( "extdata", "example.bam", package = "wavClusteR" )
> Bam <- readSortedBam(filename = filename)
> Bam
\end{Sinput}
\begin{Soutput}
GRangesList object of length 1:
[[1]] 
GRanges object with 5358 ranges and 2 metadata columns:
         seqnames               ranges strand   |
            <Rle>            <IRanges>  <Rle>   |
     [1]     chrX [24001819, 24001844]      -   |
     [2]     chrX [24001819, 24001843]      -   |
     [3]     chrX [24001834, 24001863]      -   |
     [4]     chrX [24001836, 24001865]      -   |
     [5]     chrX [24001841, 24001876]      -   |
     ...      ...                  ...    ... ...
  [5354]     chrX [24023018, 24023051]      -   |
  [5355]     chrX [24023018, 24023051]      -   |
  [5356]     chrX [24023019, 24023051]      -   |
  [5357]     chrX [24023019, 24023051]      -   |
  [5358]     chrX [24023067, 24023090]      -   |
                                         qseq          MD
                               <DNAStringSet> <character>
     [1]           CAGAGATAAAGAAGTATATTTTAAAG          26
     [2]            CAGAGATAAAGAAGTATATTTTAAG        24A0
     [3]       ATATTTTAGAGATTAAAAATATTTTATTTA        8A21
     [4]       TTTTTAAAGATTAAGAATATTTTATTTAAA     0A13A15
     [5] AAAGATTAAAAATATTTTATTTAAGCTTTTCTTCAT       24A11
     ...                                  ...         ...
  [5354]   GTTTCACAGCGTTTTGGAGGAAAAAAAAATATGT       10A23
  [5355]   GTTTCACAGCGTTTTGGAGGAAAAAAAAATATGT       10A23
  [5356]    TTTCACAGCGTTTTGGAGGAAAAAAAAATATGT        9A23
  [5357]    TTTCACAGCGTTTTGGAGGAAAAAAAAATATGT        9A23
  [5358]             CAAAGGCGCGAATGGGTTTATTTT        9A14

-------
seqinfo: 25 sequences from an unspecified genome; no seqlengths
\end{Soutput}
\end{Schunk}
%
%%--------------------------------------------------
\subsection{Extracting the informative positions used for model parameter estimation} 
%%--------------------------------------------------
%
To estimate the mixture model both mixing coefficients and density functions (components) have to be estimated from the data. To this purpose, genome-wide substitutions are first identified and filtered according to a minimum coverage value at substitutions. The minimum coverage, which should be chosen to account for variables such as sequencing depth, provides a way to select the positions used for parameter estimation. Hence, it can be used to tune the stringency of the analysis. There is no obvious theoretical justification to find the optimal minimum coverage. However since relative substitution frequencies have to be computed for parameter estimation, the minimum coverage will influence the variance of the estimate. The lowest minimum coverage used for our analysis was 10.\\
The \Rfunction{getAllSub} function identifies the genomic positions that show at least one substitution and satisfy the minimum coverage requirement. It returns a \Robject{GRanges} object specifying the genomic position, the strand, the observed substitution (e.g. "TC" implies a T in the reference genome and a C in the read), the strand-specific coverage and the number of observed substitutions at the specific position.
%
\begin{Schunk}
\begin{Sinput}
> countTable <- getAllSub( Bam, minCov = 10 )